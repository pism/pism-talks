\documentclass[hide notes,intlimits]{beamer}


\mode<presentation>
{
  \usetheme{UAFshade}
  \setbeamercovered{transparent}
}

% load packages
\usepackage{multimedia}
\usepackage{animate}
\usepackage[english]{babel}
\usepackage[latin1]{inputenc}
\usepackage[T1]{fontenc}
\usepackage{lmodern}
\usepackage[multidot]{grffile}
\usepackage[amssymb]{SIunits}


\usepackage{tikz}
\usetikzlibrary{shapes,arrows,shadows, calc}

% \usepackage{pgfpages}
% \setbeamertemplate{note page}[plain]
% \setbeameroption{show notes on second screen=right}


\definecolor{dark red}{HTML}{E41A1C}
\definecolor{dark green}{HTML}{4DAF4A}
\definecolor{dark violet}{HTML}{984EA3}
\definecolor{dark blue}{HTML}{084594}
\definecolor{dark orange}{HTML}{FF7F00}
\definecolor{white}{HTML}{FFFFFF}
\definecolor{light blue}{HTML}{377EB8}
\definecolor{light red}{HTML}{FB9A99}
\definecolor{light violet}{HTML}{CAB2D6}

\definecolor{uaf red}{HTML}{E41A1C}
\definecolor{uaf blue}{HTML}{377EB8}
\definecolor{uaf green}{HTML}{4DAF4A}
\definecolor{uaf violet}{HTML}{984EA3}
\definecolor{uaf orange}{HTML}{FF7F00}
\setbeamercolor{boxed}{fg=black,bg=uaf yellow}


% Define block styles
\tikzstyle{initialization} = [ellipse, draw, 
    text badly centered, draw=dark violet,
        % The filling: 
        top color=white, 
        bottom color=dark violet]
\tikzstyle{initialization faded} = [ellipse, draw, 
    text badly centered, draw=dark violet!50,
        % The filling: 
        top color=white, 
        bottom color=dark violet!25]
\tikzstyle{hindcast} = [ellipse, draw,
    text badly centered, rounded corners,draw=dark orange,
        % The filling: 
        top color=white, 
        bottom color=dark orange]
\tikzstyle{hindcast faded} = [ellipse, draw,
    text badly centered, rounded corners,draw=dark orange!50,
        % The filling: 
        top color=white, 
        bottom color=dark orange!25]
\tikzstyle{forecast} = [ellipse, draw,
    text badly centered, rounded corners,draw=dark blue,
        % The filling: 
        top color=white, 
        bottom color=dark blue]
\tikzstyle{forecast faded} = [ellipse, draw,
    text badly centered, rounded corners,draw=dark blue!50,
        % The filling: 
        top color=white, 
        bottom color=dark blue!50]
\tikzstyle{arrow line} = [draw, -latex']
\tikzstyle{line} = [draw]



\graphicspath{{figures/}}

\setbeamerfont{caption}{size=\scriptsize}

% code adapted from http://tex.stackexchange.com/a/11483/3954

% some parameters for customization
\def\shadowshift{3pt,-3pt}
\def\shadowradius{6pt}

\colorlet{innercolor}{black!60}
\colorlet{outercolor}{gray!05}

% this draws a shadow under a rectangle node
\newcommand\drawshadow[1]{
    \begin{pgfonlayer}{shadow}
        \shade[outercolor,inner color=innercolor,outer color=outercolor] ($(#1.south west)+(\shadowshift)+(\shadowradius/2,\shadowradius/2)$) circle (\shadowradius);
        \shade[outercolor,inner color=innercolor,outer color=outercolor] ($(#1.north west)+(\shadowshift)+(\shadowradius/2,-\shadowradius/2)$) circle (\shadowradius);
        \shade[outercolor,inner color=innercolor,outer color=outercolor] ($(#1.south east)+(\shadowshift)+(-\shadowradius/2,\shadowradius/2)$) circle (\shadowradius);
        \shade[outercolor,inner color=innercolor,outer color=outercolor] ($(#1.north east)+(\shadowshift)+(-\shadowradius/2,-\shadowradius/2)$) circle (\shadowradius);
        \shade[top color=innercolor,bottom color=outercolor] ($(#1.south west)+(\shadowshift)+(\shadowradius/2,-\shadowradius/2)$) rectangle ($(#1.south east)+(\shadowshift)+(-\shadowradius/2,\shadowradius/2)$);
        \shade[left color=innercolor,right color=outercolor] ($(#1.south east)+(\shadowshift)+(-\shadowradius/2,\shadowradius/2)$) rectangle ($(#1.north east)+(\shadowshift)+(\shadowradius/2,-\shadowradius/2)$);
        \shade[bottom color=innercolor,top color=outercolor] ($(#1.north west)+(\shadowshift)+(\shadowradius/2,-\shadowradius/2)$) rectangle ($(#1.north east)+(\shadowshift)+(-\shadowradius/2,\shadowradius/2)$);
        \shade[outercolor,right color=innercolor,left color=outercolor] ($(#1.south west)+(\shadowshift)+(-\shadowradius/2,\shadowradius/2)$) rectangle ($(#1.north west)+(\shadowshift)+(\shadowradius/2,-\shadowradius/2)$);
        \filldraw ($(#1.south west)+(\shadowshift)+(\shadowradius/2,\shadowradius/2)$) rectangle ($(#1.north east)+(\shadowshift)-(\shadowradius/2,\shadowradius/2)$);
    \end{pgfonlayer}
}

% create a shadow layer, so that we don't need to worry about overdrawing other things
\pgfdeclarelayer{shadow} 
\pgfsetlayers{shadow,main}

\newsavebox\mybox
\newlength\mylen

\newcommand\shadowimage[2][]{%
\setbox0=\hbox{\includegraphics[#1]{#2}}
\setlength\mylen{\wd0}
\ifnum\mylen<\ht0
\setlength\mylen{\ht0}
\fi
\divide \mylen by 120
\def\shadowshift{\mylen,-\mylen}
\def\shadowradius{\the\dimexpr\mylen+\mylen+\mylen\relax}
\begin{tikzpicture}
\node[anchor=south west,inner sep=0] (image) at (0,0) {\includegraphics[#1]{#2}};
\drawshadow{image}
\end{tikzpicture}}

\newcommand\shadowimagec[3][]{%
\setbox0=\hbox{\includegraphics<#1>[#2]{#3}}
\setlength\mylen{\wd0}
\ifnum\mylen<\ht0
\setlength\mylen{\ht0}
\fi
\divide \mylen by 120
\def\shadowshift{\mylen,-\mylen}
\def\shadowradius{\the\dimexpr\mylen+\mylen+\mylen\relax}
\begin{tikzpicture}
\node[anchor=south west,inner sep=0] (image) at (0,0) {\includegraphics<#1>[#2]{#3}};
\drawshadow{image}
\end{tikzpicture}}


\newenvironment{transbox}[1][]{%
\begin{tikzpicture}
\node[drop shadow,rounded corners,text width=\textwidth,fill=white, fill opacity=#1,text opacity=1] \bgroup
}{
\egroup;\end{tikzpicture}} 

\newenvironment{transbox-tight}{%
\begin{tikzpicture}
\node[drop shadow,rounded corners,fill=uaf yellow, fill opacity=0.75,text opacity=1] \bgroup
}{
\egroup;\end{tikzpicture}} 


% title page
\title[] % (optional, use only with long paper titles)
{Glaciers: The Biggest Losers}



\author[Aschwanden] % (optional, use only with lots of authors)
{Andy Aschwanden}
% - Give the names in the same order as the appear in the paper.
% - Use the \inst{?} command only if the authors have different
%   affiliation.

\institute[Geophysical Institute] % (optional, but mostly needed)
{Geophysical Institute}
% - Use the \inst command only if there are several affiliations.
% - Keep it simple, no one is interested in your street address.


\date{}

\begin{document}

% define what is shown at the beginning of each section
\AtBeginSection[]
{
  \begin{frame}<handout:0>
    \frametitle{Outline}
   \tableofcontents[currentsection,subsectionstyle=hide/hide/hide]
  \end{frame}
}

% define what is shown at the beginning of each subsection
\AtBeginSubsection[]
{
 \begin{frame}<beamer>
  \frametitle{Outline}
   \tableofcontents[currentsection,currentsubsection]
 \end{frame}
}


\setbeamertemplate{background canvas}
  {
     \tikz{\node[inner sep=0pt,opacity=1.] {\includegraphics[width=\paperwidth]{galenstock_bg}};}
}


% insert titlepage
\begin{frame}
  \titlepage
\end{frame}

\setbeamertemplate{background canvas}
  {
} 


\setbeamertemplate{background canvas}
  {
     \tikz{\node[inner sep=0pt,opacity=1.] {\includegraphics[width=\paperwidth]{uri-collage}};}
}

\begin{frame}[plain]
  \note[item]{I grew up in Switzerland, in the Kanton Uri---a Kanton is the equivalent to a State}
  \note[item]{True to the stereotype, we have many cows in Uri and produce a lot of good cheese}
  \note[item]{Last year we had on bear roaming through Uri, which gave it some Alaska-feel}
  \note[item]{According to legend, William Tell was born in Uri, who freed us from the Austrian oppression in 1291}
\end{frame}

\setbeamertemplate{background canvas}
{
  \tikz{\node[inner sep=0pt,opacity=1.] {\includegraphics[width=\paperwidth]{andy-ski}};}
}

\begin{frame}[plain]
  \note[item]{I grew up on skis, I learned to downhill ski right after learning to walk}
  \note[item]{}
\end{frame}



\setbeamertemplate{background canvas}
{
  \tikz{\node[inner sep=0pt,opacity=1.] {\includegraphics[width=\paperwidth]{ski-climb}};}
}

\begin{frame}[plain]
  \note[item]{When I was 20, I became tired of waiting for the next gondola}
  \note[item]{and started with ski mountaineering}
\end{frame}

\setbeamertemplate{background canvas}
{
  \tikz{\node[inner sep=0pt,opacity=1.] {\includegraphics[width=\paperwidth]{gurschengletscher-l}};}
}

\begin{frame}[plain]
\end{frame}

\setbeamertemplate{background canvas}
{
  \tikz{\node[inner sep=0pt,opacity=1.] {\includegraphics[width=\paperwidth]{gurschengletscher}};}
}

\begin{frame}[plain]
\end{frame}
  
  
\setbeamertemplate{background canvas}
{
} 


\begin{frame}
  \begin{itemize}
    \item Presently, 10 percent of land area on Earth is covered with glacial ice, including glaciers, ice caps, and the ice sheets of Greenland and Antarctica. Glacierized areas cover over 15 million square kilometers (5.8 million square miles).
    \item Glaciers store about 75 percent of the world's fresh water.
    \item During the maximum point of the last ice age, glaciers covered about 32 percent of the total land area.
\item size of AK: 663,268 sq mi (1,717,856 km2), glaciers: 87,000 sq km
  \end{itemize}
\end{frame}

\begin{frame}
\begin{itemize}
\item what do we use glaciers for?
\item contain 75\% of the world's fresh water
\item electricity, hydrodam, summer
\end{itemize}
\end{frame}

\begin{frame}{Glaciers Flow}
\begin{itemize}
\item what do we use glaciers for?
\item contain 75\% of the world's fresh water
\item electricity, hydrodam, summer
\end{itemize}
\end{frame}


\begin{frame}{Steigletscher 1994 / 2006}
      \begin{figure}
        \includegraphics<1>[width=\textwidth]{steigletscher-repeat-1994}
        \includegraphics<2>[width=\textwidth]{steigletscher-repeat-2006}
      \end{figure}
\end{frame}



\begin{frame}{Marine Ice Sheet Instability Hypothesis}
  \begin{columns}[c]
    \begin{column}{.65\linewidth}
      \begin{figure}
        \includegraphics<1>[height=8cm]{ant-marine}
      \end{figure}
    \end{column}
    \begin{column}{.38\linewidth}
      \begin{itemize}
      \item blue colors: below sea level
      \end{itemize}
    \end{column}
  \end{columns}
\end{frame}


\end{document}
