\documentclass[hide notes,intlimits]{beamer}


\mode<presentation>
{
  \usetheme[footline]{UAFshade}
  \setbeamercovered{transparent}
}

% load packages
\usepackage{multimedia}
\usepackage{animate}
\usepackage[english]{babel}
\usepackage[latin1]{inputenc}
\usepackage[T1]{fontenc}
\usepackage{lmodern}
\usepackage[multidot]{grffile}

\usepackage{tikz}
\usetikzlibrary{shapes,arrows,shadows, calc}

% \usepackage{pgfpages}
% \setbeamertemplate{note page}[plain]
% \setbeameroption{show notes on second screen=right}


\definecolor{dark red}{HTML}{E41A1C}
\definecolor{dark green}{HTML}{4DAF4A}
\definecolor{dark violet}{HTML}{984EA3}
\definecolor{dark blue}{HTML}{084594}
\definecolor{dark orange}{HTML}{FF7F00}
\definecolor{light blue}{HTML}{377EB8}
\definecolor{light red}{HTML}{FB9A99}
\definecolor{light violet}{HTML}{CAB2D6}

\definecolor{uaf red}{HTML}{E41A1C}
\definecolor{uaf blue}{HTML}{377EB8}
\definecolor{uaf green}{HTML}{4DAF4A}
\definecolor{uaf violet}{HTML}{984EA3}
\definecolor{uaf orange}{HTML}{FF7F00}
\setbeamercolor{boxed}{fg=black,bg=uaf yellow}

\graphicspath{{figures/}}

\setbeamerfont{caption}{size=\scriptsize}

% code adapted from http://tex.stackexchange.com/a/11483/3954

% some parameters for customization
\def\shadowshift{3pt,-3pt}
\def\shadowradius{6pt}

\colorlet{innercolor}{black!60}
\colorlet{outercolor}{gray!05}

% this draws a shadow under a rectangle node
\newcommand\drawshadow[1]{
    \begin{pgfonlayer}{shadow}
        \shade[outercolor,inner color=innercolor,outer color=outercolor] ($(#1.south west)+(\shadowshift)+(\shadowradius/2,\shadowradius/2)$) circle (\shadowradius);
        \shade[outercolor,inner color=innercolor,outer color=outercolor] ($(#1.north west)+(\shadowshift)+(\shadowradius/2,-\shadowradius/2)$) circle (\shadowradius);
        \shade[outercolor,inner color=innercolor,outer color=outercolor] ($(#1.south east)+(\shadowshift)+(-\shadowradius/2,\shadowradius/2)$) circle (\shadowradius);
        \shade[outercolor,inner color=innercolor,outer color=outercolor] ($(#1.north east)+(\shadowshift)+(-\shadowradius/2,-\shadowradius/2)$) circle (\shadowradius);
        \shade[top color=innercolor,bottom color=outercolor] ($(#1.south west)+(\shadowshift)+(\shadowradius/2,-\shadowradius/2)$) rectangle ($(#1.south east)+(\shadowshift)+(-\shadowradius/2,\shadowradius/2)$);
        \shade[left color=innercolor,right color=outercolor] ($(#1.south east)+(\shadowshift)+(-\shadowradius/2,\shadowradius/2)$) rectangle ($(#1.north east)+(\shadowshift)+(\shadowradius/2,-\shadowradius/2)$);
        \shade[bottom color=innercolor,top color=outercolor] ($(#1.north west)+(\shadowshift)+(\shadowradius/2,-\shadowradius/2)$) rectangle ($(#1.north east)+(\shadowshift)+(-\shadowradius/2,\shadowradius/2)$);
        \shade[outercolor,right color=innercolor,left color=outercolor] ($(#1.south west)+(\shadowshift)+(-\shadowradius/2,\shadowradius/2)$) rectangle ($(#1.north west)+(\shadowshift)+(\shadowradius/2,-\shadowradius/2)$);
        \filldraw ($(#1.south west)+(\shadowshift)+(\shadowradius/2,\shadowradius/2)$) rectangle ($(#1.north east)+(\shadowshift)-(\shadowradius/2,\shadowradius/2)$);
    \end{pgfonlayer}
}

% create a shadow layer, so that we don't need to worry about overdrawing other things
\pgfdeclarelayer{shadow} 
\pgfsetlayers{shadow,main}

\newsavebox\mybox
\newlength\mylen

\newcommand\shadowimage[2][]{%
\setbox0=\hbox{\includegraphics[#1]{#2}}
\setlength\mylen{\wd0}
\ifnum\mylen<\ht0
\setlength\mylen{\ht0}
\fi
\divide \mylen by 120
\def\shadowshift{\mylen,-\mylen}
\def\shadowradius{\the\dimexpr\mylen+\mylen+\mylen\relax}
\begin{tikzpicture}
\node[anchor=south west,inner sep=0] (image) at (0,0) {\includegraphics[#1]{#2}};
\drawshadow{image}
\end{tikzpicture}}

\newcommand\shadowimagec[3][]{%
\setbox0=\hbox{\includegraphics<#1>[#2]{#3}}
\setlength\mylen{\wd0}
\ifnum\mylen<\ht0
\setlength\mylen{\ht0}
\fi
\divide \mylen by 120
\def\shadowshift{\mylen,-\mylen}
\def\shadowradius{\the\dimexpr\mylen+\mylen+\mylen\relax}
\begin{tikzpicture}
\node[anchor=south west,inner sep=0] (image) at (0,0) {\includegraphics<#1>[#2]{#3}};
\drawshadow{image}
\end{tikzpicture}}


\newenvironment{transbox}[1][]{%
\begin{tikzpicture}
\node[drop shadow,rounded corners,text width=\textwidth,fill=white, fill opacity=#1,text opacity=1] \bgroup
}{
\egroup;\end{tikzpicture}} 

\newenvironment{transbox-tight}{%
\begin{tikzpicture}
\node[drop shadow,rounded corners,fill=uaf yellow, fill opacity=0.75,text opacity=1] \bgroup
}{
\egroup;\end{tikzpicture}} 


% title page
\title[] % (optional, use only with long paper titles)
{Chronicles of the Greenland Ice Sheet}



\author[Aschwanden] % (optional, use only with lots of authors)
{Andy Aschwanden}
% - Give the names in the same order as the appear in the paper.
% - Use the \inst{?} command only if the authors have different
%   affiliation.

\institute[Geophysical Institute] % (optional, but mostly needed)
{\textcolor{white}{Geophysical Institute}}
% - Use the \inst command only if there are several affiliations.
% - Keep it simple, no one is interested in your street address.

\date{}
\titlegraphic{\vskip-0.5cm\shadowimage[width=\textwidth]{gris-nw-speed-exp-600m}}


\begin{document}

% define what is shown at the beginning of each section
\AtBeginSection[]
{
  \begin{frame}<handout:0>
    \frametitle{Outline}
   \tableofcontents[currentsection,subsectionstyle=hide/hide/hide]
  \end{frame}
}

% define what is shown at the beginning of each subsection
\AtBeginSubsection[]
{
 \begin{frame}<beamer>
  \frametitle{Outline}
   \tableofcontents[currentsection,currentsubsection]
 \end{frame}
}



\setbeamertemplate{background canvas}
  {
     \tikz{\node[inner sep=0pt,opacity=1.0] {\includegraphics[width=\paperwidth]{uaf_beamer_shade_bg}};}
} 


% insert titlepage
\begin{frame}
  \titlepage
\end{frame}

\setbeamertemplate{background canvas}
{
%
} 

\begin{frame}{Greenland}
\vspace{-1em}
    \begin{figure}
      \includegraphics<1>[width=.85\textwidth]{greenland-obs-overview_rotated-01}
      \includegraphics<2>[width=.85\textwidth]{greenland-obs-basal-overview_rotated-01}
   \end{figure}
\end{frame}

\begin{frame}{Greenland}
  \begin{columns}
    \column[T]{5cm}
    \begin{figure}
      \includegraphics<1>[height=.75\textheight]{greenland-obs-basal-overview-mo14}
      {\footnotesize 
      Observed speeds (Nagler et al, 2015)}
    \end{figure}
    \column[T]{5cm}
    \begin{figure}
      \includegraphics<1>[height=.75\textheight]{greenland-obs-overview}
      {\small 
      PISM simulation for SeaRISE}
    \end{figure}
  \end{columns}
\end{frame}


\begin{frame}{2016}
    \begin{figure}
      \includegraphics<1>[width=.8\textwidth]{greenland-overview-3}
      \footnotesize 
      \caption{}
    \end{figure}
\end{frame}


\begin{frame}{IPCC AR5}
  \begin{columns}
    \column[T]{5cm}
    \begin{figure}
      \includegraphics<1>[height=.70\textheight]{gris-overview-speed-obs}
      \footnotesize 
      \caption{Observed speeds (Nagler et al, 2015)}
    \end{figure}
    \column[T]{5cm}
    \begin{figure}
      \includegraphics<1>[height=.70\textheight]{gris-overview-speed-searise}
      \small 
      \caption{PISM simulation for SeaRISE}
    \end{figure}
  \end{columns}
\end{frame}

\begin{frame}{2016}
  \begin{columns}
    \column[T]{5cm}
    \begin{figure}
      \includegraphics<1>[height=.70\textheight]{gris-overview-speed-obs}
      \footnotesize 
      \caption{Observed speeds (Nagler et al, 2015)}
    \end{figure}
    \column[T]{5cm}
    \begin{figure}
      \includegraphics<1>[height=.70\textheight]{gris-overview-speed-exp}
      \small 
      \footnotesize 
      \caption{PISM simulation (Aschwanden, Fahnestock, Truffer, 2016, Nature Comm.)}
    \end{figure}
  \end{columns}
\end{frame}

\begin{frame}{IPCC AR5}
    \begin{figure}
      \includegraphics<1>[width=.8\textwidth]{nw-gris-600m_obs}
      \footnotesize 
      \caption{Observed speeds (Nagler et al, 2015)}
    \end{figure}
    \begin{figure}
      \includegraphics<1>[width=.8\textwidth]{nw-gris-600m_searise}
      \footnotesize
      \caption{PISM simulation for SeaRISE}
    \end{figure}
\end{frame}

\begin{frame}{2016}
    \begin{figure}
      \includegraphics<1>[width=.8\textwidth]{nw-gris-600m_obs}
      \footnotesize 
      \caption{Observed speeds (Nagler et al, 2015)}
    \end{figure}
    \begin{figure}
      \includegraphics<1>[width=.8\textwidth]{nw-gris-600m_best_v1}
      \footnotesize
      \caption{PISM simulation (Aschwanden, Fahnestock, Truffer, 2016, Nature Comm.)}
    \end{figure}
\end{frame}

\begin{frame}{Projections}
If we want to project the evolution of our ice sheets, we first need to look at the past

Are the model illustrations compatible with our understanding of since-LGM change from observations?
\end{frame}


\end{document}
