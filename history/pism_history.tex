\documentclass[11pt]{article}
\usepackage[margin=1in]{geometry}
\usepackage[OT1,T1]{fontenc}
\usepackage[utf8x]{inputenc}



\begin{document}

\title{The Parallel Ice Sheet Model (PISM): The early years}
\maketitle

\section*{before 2001}

This document on the history of PISM got its start when Craig Lingle wrote the following in an October 2015 email to Ed Bueler:

``\emph{PISM} actually has a fairly long background, extending back into the “deep glaciological past,” prior to when Vera Veronina (a Russian emigre who was working for me on analysis of Antarctic satellite radar altimeter data) decided to leave Fairbanks and accept an offer at U.C. Boulder (she was a warm weather woman), so she introduced me to Elena Troshina (now Suleimani, Ph.D., also a Russian emigre, who had completed her M.S. in marine sciences based on her tsunami modeling). Elena said she wanted to work for me.
I looked at her thesis, immediately recognized her mathematical and numerical modeling abilities, and put her to work on improving the (nominally Antarctic) ice sheet model I had been carrying around under one arm (so to speak), through 3 moves, ever since I developed it (based on Mary-Ann (M.A.W.) Mahaffey’s equations and numerical method that she had published in the Journal of Geophysical Research a few years earlier (with minimal detail). I tested it against the analytical solution for a circular ice cap on a flat bed with constant accumulation, and satisfied myself that it was accurate. Then, I applied it to the 20 K BP CLIMAP reconstruction of the Antarctic ice sheet (with the idea of modeling forward from there to the present), and satisfied myself that it was stable. Then, I set it aside, because it was… well.. too complicated to address the “grounding line dynamics” problem I had in mind for my Ph.D. thesis.
Many years passed. When Elena began working for me, I dug it up — still carefully preserved, in an archeological sense — and asked Elena to transform it from Fortran 4 to Fortran 90. She did. Then, I gave her the equations for transforming the model equations to forms with a stretched vertical coordinate. She was properly skeptical, and checked all my equation. Then, she did. Then, I gave her the 3-D temperature equation expressed with a stretched vertical coordinate, and asked her to incorporate it. She again checked the equations, and did so. But, it wasn’t stable, for the reason you quickly identified when I first introduced you to Elena, and we talked about that.
Anyway---all that work by Elena led to the version of the ice sheet model that you saw when you first expressed interest in it, and the published paper [C. Lingle and E. Troshina, 1998. Relative magnitudes of shear and longitudinal strain rates in the inland Antarctic ice sheet, and response to increasing accumulation, Ann. Glaciol. 27, pp.187-193] that enabled us to apply [in 2001] to NASA for model-development funding [namely grant NAG5-11371].|''


\section*{2001: team forms}

At this point, because of Craig's initiative to expand it, the project started including a lot more people.

In 2001 Craig Lingle attended a talk on heat equations on manifolds given by Ed Bueler. Lingle suggested that Bueler might want to work on modeling heat flow in glaciers. This became a short course by Craig on glaciological basics, with Bueler, Latrice Bowman, Jed Brown, and Dave Covey in attendance at various points. Craig optimistically led a proposal to NASA, with Bueler and Covey as Co-Is, which was funded as NAG5-11371, as a modeling component (sub-grant) of the U Kansas “Polar Radar for Ice Sheet Measurements” (PRISM) project and NASA grant, to build an Antarctic model which added thermomechanical coupling and ice shelf dynamics to the existing Fortran model.

Bueler was interested in understanding numerical models by checking them against exact predictions (solutions) of the differential equations. This became a paper [E. Bueler, C. Lingle, J. Kallen-Brown, D. Covey, L. Bowman, 2005. Exact solutions and verification of numerical models for isothermal ice sheets, J. Glaciol. 51 (173), 291–306], but the first submission in 2003 did not get accepted. This work used Matlab scripts, instead of the Fortran code, but it provided tests which, around the time of the re-submission, were used to check the code that would become PISM. In this period, Latrice was the first of Bueler's graduate students to work on ice flow, with MS Math based on this work in 2002.

\section*{2003: PETSc and C\ldots{}and PISM}

In 2002 Jed Brown became involved with PISM as an undergraduate, working for a while with the Fortran code from Craig and Elena. Around 2003/04 Brown came into Bueler's office and said, essentially, that there was this nice library that would allow us to work in parallel at a higher conceptual level, namely PETSc. And that we should switch to C-plus-plus so that the code could be more modular. This suggestion was not fully appreciated by Bueler, but it was fully accepted. The Fortran code was dropped, object classes were defined, and the isothermal SIA model, with under-development thermomechanical-coupling code, rebuilt based on collaboration between Brown and Bueler.

Three goals for major additions then followed, in a period when Brown became very familiar with PETSc and Bueler finally learned C:

Bueler led the effort to add thermocoupling to the SIA, with Brown and Lingle assistance, and emphasizing exact solutions to check. This became E. Bueler, J. Brown, and C. Lingle, (2007). Exact solutions to the thermomechanically coupled shallow-ice approximation: effective tools for verification, J. Glaciol. 53 (182), 499–516.
Because Brown was now an MS student in math, Bueler suggested that Brown's MS project be the addition of, and testing of, a SSA solver in PISM. This led to a successful August 2006 MS project defense. At that time the model had this (Bueler's suggestion) name: the C-plus-plus Object-oriented Multi-Modal, Verifiable Numerical Ice Sheet Model, a.k.a. COMMVNISM.
NetCDF was adopted as the input/output format. Before this, PETSc binary files were used. (This fast format lacks included and standardized metadata.)
As this work was finishing, three things became clear: the multi-modal aspect was not actually working, Brown would be graduating and leaving for a PhD in Zurich, and another proposal would soon need to be written. Brown renamed the model one day around this time—with no opposition—to the less cumbersome and better-suited-to-a-proposal name of “Parallel Ice Sheet Model”, PISM.

\section*{2006: PISM goes public}

In September 2006 PISM was for the first time hosted publicly, on GNA with a GNU General Public License. We benefited greatly from using SVN and having free GNA hosting, even though we eventually moved happily to git and github.

\section*{2007: PISM gets ice streams}

The next idea, circa mid-2007, that went into PISM was that the SSA should be solved everywhere. This is because, in a Coulomb or near-Coulomb basal drag regime, the solution simply returns zero sliding where the base is sufficiently strong. Solving everywhere thus defines the ice stream regions organically. This idea arose because Bueler actually read C. Schoof's isothermal paper [C. Schoof (2006). A variational approach to ice stream flow, J. Fluid Mech. 556, 227–251], and realized this made just as much sense in a thermocoupled context.

Solving the SIA everywhere would do no harm because in low-angle streams and shelves the SIA produces low velocities. Furthermore, a convex combination of two reasonable stress balance solutions (i.e. SIA+SSA) was reasonable.

Almost as important, in driving the creation of the solve-SSA-everywhere model, was the failure of PISM, and every other model, to produce anything sensible from the ISMIP-HEINO modeling assumptions/requirements and boundary conditions. The issue seen in that project is that, in essence, there is no way to switch sliding on or off, in a physically-based thermomechanically-coupled manner, entirely within the SIA paradigm. One needs to balance transitions in boundary shear stress with membrane stresses within the ice.

In other words, the SIA is a good model, but not of sliding (or ice shelves, for that matter).

These ideas, and Jed's work on the PISM SSA implementation, led to actually having ice streams in the model for the right reasons by 2007. The paper E. Bueler and J. Brown, (2009). Shallow shelf approximation as a “sliding law” in a thermomechanically coupled ice sheet model, J. Geophys. Res. 114 (F3) was the result. This paper turned out to be the core of PISM, and it is the most-cited of the PISM-related papers.

\section*{2008: new team and PIK collaboration}

A new proposal to a NASA Modeling, Analysis, and Prediction call, based on a team of Bueler, Khroulev, mathematician David Maxwell, and glaciologists Martin Truffer and Regine Hock, was funded as NNX09AJ38C. To be continued …

The new proposal made sense because Constantine Khroulev was finishing his MS Math in Fall 2007 (on unrelated math topics). That is, he became recruitable to the project. Constantine started in spring 2008 and was soon inside PISM. He produced startlingly-complete Antarctic results in poster form in Fall 2008 at the WAIS conference. But PISM was distincly lacking in ability to move the calving front and the grounding line.

Without needing excess tuning based on generally-unavailable data (i.e. inversion of measured surface velocities) the results for surface velocity have the right look. Indeed, by early 2009 we saw that nearly-untuned results for Greenland matched observations reasonably well.

Fall 2008 involved another big change: Anders Levermann and students (Maria Martin and Ricarda Winkelmann) came to Fairbanks to propose a collaboration in which they would add what PISM needed. To be continued …

\section*{2009}
And another: Andy Aschwanden was hired as an ARSC PostDoc in Spring 2009.

2011: PISM goes viral

\end{document}
